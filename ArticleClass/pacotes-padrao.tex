%Preâmbulo
\usepackage[brazil]{babel}%Idioma
\usepackage[utf8]{inputenc}%Inserir acentos 
\usepackage[T1]{fontenc}%Tipo de fonte usada na compilação
%\usepackage[osf]{Alegreya,AlegreyaSans}%Fontes Alegreya
\usepackage{microtype}%Microtipografia
\usepackage{hyphenat}%Habilita/desabilita hifenização
\usepackage{soul}%Facilidades para manipualar formatação de FONTES
\usepackage{calc}%Medir margens do pacote Geometry
\usepackage[a4paper,left=2.0cm, right=2.0cm, top=2.0cm, bottom=2.0cm]{geometry}
\usepackage{graphicx}%inserir imagens
\usepackage[x11names]{xcolor}%Texto colorido
\usepackage{pdfpages}%Inserir páginas PDF
\usepackage{tcolorbox}%Caixa de texto >>>> Ver abaixo
\tcbuselibrary{skins}% Obrigatório para formatação específica da tcolorbox
\usepackage{multicol}%Criar multicolunas sem \tabular
\usepackage{url}%Para disponibilizar o endereço do site
\usepackage[hidelinks]{hyperref}%Remover caixa vermelha no TOC
\usepackage{lipsum}%para gerar lipsum
\usepackage{alltt}%Para escrever versos sem precisar indicar “\\”
\usepackage{verbatim}%Citar códigos Latex
\usepackage{smartdiagram}%Diagramas
\usepackage{tikz}%Desenhos impossíveis
\usepackage{tabularx}%Tabelas mais fáceis
\usepackage{setspace}%Mudar espaçamento entre linhas

%Referências e Citações
\usepackage[alf,abnt-repeated-title-omit=yes,abnt-emphasize-bf,ebnt-etal-list=0]{abntex2cite}
\citebrackets()
%Para citação direta, usar \cite{entrada do arq. referencias}
%Para citação indireta, usar \citeonline{entrada do arq. referencias}

%Para citações longas
\renewenvironment{quote}
{\small\list{}{\rightmargin=0cm \leftmargin=4cm}%
	\item\relax}
{\endlist}